\subsection{Introduction}
In this section the workflow and results of the Inspection Evaluation are presented and discussed.

The task asked us to act as experts to evaluate the quality and usability of a website assigned by the evaluators of the course. What follows is an introduction to the content of the website, then the heuristics used are presented, with the agreement on the scale and scores used. A comment on the results is finally presented.


% TODO Talk about the general method and steps followed - How the study was conducted

\subsection{Overview of the Website}
The website assigned for the June/July call is \underline{\url{www.unicef.org}}.

UNICEF, or the United Nations Children’s Fund, is a United Nations agency responsible for providing humanitarian and developmental aid to children worldwide. Founded in 1946, UNICEF’s mission is to advocate for the protection of children’s rights, to help meet their basic needs, and to expand their opportunities to reach their full potential. UNICEF operates in over 190 countries and territories, focusing on areas such as child protection, education, immunization, healthcare, nutrition, and emergency relief.

The website presents a virtually infinite amount of information about the activity of the Agency: it presents News and Stories regarding the initiatives taken; Programs addressing various field, Donations which fund the organization; Resources to get involved as a volunteer or worker.

It should be noted that various sub-websites (such as the one to file an application for a career within UNICEF, the Donation website etc.) are presented throughout the main website. Such websites were ignored during the evaluation. The group decided to focus on websites that would provide knowledge and content regarding the Agency activity, rather than evaluating applications which represent a really small part of the entire website.

\subsection{Heuristics}
\textbf{Usability inspection} is the name for a set of methods where an evaluator inspects a user interface.
The evaluation is enabled by the application of the so called Heuristic-driven evaluation technique, a set of principles and checklists that focus on various areas to help discovering potential flaws and issues.


\subsubsection{Nielsen}
Nielsen heuristics, formulated by Jakob Nielsen, are a set of ten general principles for user interface design, introduced in 1989, updated in 1994, finalized in 2005. They are widely used as guidelines for evaluating and improving the usability of a system.
\begin{enumerate}
	\item \textbf{N1} Visibility of system status;
	\item \textbf{N2} Match between system and the real world;
	\item \textbf{N3} User control and freedom;
	\item \textbf{N4} Consistency and standards;
	\item \textbf{N5} Error prevention;
	\item \textbf{N6} Recognition rather than recall;
	\item \textbf{N7} Flexibility and efficiency of use;
	\item \textbf{N8} Aesthetic and minimalist design;
	\item \textbf{N9} Help users recognize, diagnose and recover from errors;
	\item \textbf{N10} Help and documentation.
\end{enumerate}

\subsubsection{MiLe (Milan Lugano)}
The MiLE heuristics are an extension of Nielsen’s original heuristics, tailored to address specific issues related to web usability. Each heuristics tries to answer given questions (that are shown in the following list). Only a subset of the 40+ heuristics has been used.
\begin{enumerate}
	\item \textbf{Content subset}
		\begin{enumerate}
			\item \textbf{MC1} Information overload - Is the information in a page too much or too little?
			\item \textbf{MC2} Consistency of page content structure - Do pages that present topics of the same category have the same types of elements?
			\item \textbf{MC3} Contextualized information - Do the pages include information that help users understand where they are?
			\item \textbf{MC4} Content organization (hierarchy) - Is the hierarchical organization of topics appropriate for the topics relevance?
		\end{enumerate}

	\item \textbf{Navigation/Interaction subset}
		\begin{enumerate}
			\item \textbf{MN1} Interaction consistency - Do pages of the same type have the same navigation links and interaction capability?
			\item \textbf{MN2} Group navigation-1 - It is easy to navigate from/among groups of “items”, and within the items?
			\item \textbf{MN3} Group navigation-2 - Do menus create Cognitive Overload?
			\item \textbf{MN4} Structural navigation - Is it easy to navigate among the “components” (“parts”) of a topic?
			\item \textbf{MN5} Semantic navigation - Is it easy to navigate from a topic to a related one (in both directions)?
			\item \textbf{MN6} Landmarks - Are “Landmarks” effective for the user to reach the “key” (most relevant) parts of the web site?
		\end{enumerate}

	\item \textbf{Presentation subset}
		\begin{enumerate}
			\item \textbf{MP1} Text layout - Is the text readable? Is the font size appropriate?
			\item \textbf{MP2} Interaction placeholders-semiotics - Interactive elements are “intuitive”?
			\item \textbf{MP3} Interaction placeholders-consistency - Textual or visual labels of interactive elements are consistent in terms of wording, shape, color, position, etc.
			\item \textbf{MP4} Consistency of Visual Elements - In pages of the same type do visual elements have the same visual properties?
			\item \textbf{MP5} Hierarchy-1 - Is the on-screen allocation of contents within a page appropriate for their relevance?
			\item \textbf{MP6} Hierarchy-2 - Is the on-screen allocation of visual elements appropriate for their relevance?
			\item \textbf{MP7} Spatial allocation-1 - Are “Semantically related” elements close to each other?
			\item \textbf{MP8} Spatial allocation-2 - Are "Semantically distant” element placed distant from each other?
			\item \textbf{MP9} Consistency of Page Spatial Structure - Do pages of the same type have the same spatial organization for the various visual elements?
		\end{enumerate}
\end{enumerate}

\subsection{Metrics used}
% TODO present the 1-5 scale
Provide visual representations

\subsection{Final scores}

\subsection{Data Analysis - Aggregate}
PROVIDE AGGREGATED DATA, e.g., MEAN VALUES FOR ALL Heuristics,
(MEAN) SCORE BY DIMENSIONS (e.g., for CONTENT HEURISTICS,
NAVIGATION HEURISTICS, ect.)

\subsection{Screenshots}

\subsection{Conclusions}
