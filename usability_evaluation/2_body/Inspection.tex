\subsection{Introduction}
In this section the workflow and results of the Inspection Evaluation are presented and discussed.

The task asked to act as experts to evaluate the quality and usability of a website assigned by the evaluators of the course: \url{www.unicef.org}. What follows is an introduction to the content of the website, then the heuristics used are presented, with the agreement on the scale and scores used. A detailed description of the results is finally presented.


% TODO Talk about the general method and steps followed - How the study was conducted

\subsection{Overview of the Website}
The website assigned for the June/July call is \url{www.unicef.org}.

UNICEF, or the United Nations Children’s Fund, is a United Nations agency responsible for providing humanitarian and developmental aid to children worldwide. Founded in 1946, UNICEF’s mission is to advocate for the protection of children’s rights, to help meet their basic needs, and to expand their opportunities to reach their full potential. UNICEF operates in over 190 countries and territories, focusing on areas such as child protection, education, immunization, healthcare, nutrition, and emergency relief.

The website presents a virtually infinite amount of information about the activity of the Agency: it presents News and Stories regarding the initiatives taken; Programs addressing various field, Donations which fund the organization; Resources to get involved as a volunteer or worker.

It should be noted that various sub-websites (such as the one to file an application for a career within UNICEF, the Donation website etc.) are presented throughout the main website. Such websites were ignored during the evaluation. The group decided to focus on websites that would provide knowledge and content regarding the Agency activity, rather than evaluating applications which represent a really small part of the entire website.

\subsection{Heuristics}
The evaluation is enabled by the application of the so called Heuristic-driven evaluation, a set of principles that focus on various areas to help discovering potential flaws and issues within the assigned website.


\subsubsection{Nielsen}
% TODO short description of nielsen heur
\begin{enumerate}
	\item \textbf{N1} Visibility of system status;
	\item \textbf{N2} Match between system and the real world;
	\item \textbf{N3} User control and freedom;
	\item \textbf{N4} Consistency and standards;
	\item \textbf{N5} Error prevention;
	\item \textbf{N6} Recognition rather than recall;
	\item \textbf{N7} Flexibility and efficiency of use;
	\item \textbf{N8} Aesthetic and minimalist design;
	\item \textbf{N9} Help users recognize, diagnose and recover from errors;
	\item \textbf{N10} Help and documentation.
\end{enumerate}

\subsubsection{MiLe (Milan Lugano)}
% TODO short description
% TODO divide into the three categories


\subsection{Metrics used}
% TODO present the 1-5 scale
Provide visual representations

\subsection{Final scores}

\subsection{Data Analysis - Aggregate}
PROVIDE AGGREGATED DATA, e.g., MEAN VALUES FOR ALL Heuristics,
(MEAN) SCORE BY DIMENSIONS (e.g., for CONTENT HEURISTICS,
NAVIGATION HEURISTICS, ect.)

\subsection{Screenshots}

\subsection{Conclusions}
