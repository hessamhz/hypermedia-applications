\section{Comparison of the results}
To conclude this document we are going to compare the results obtained from the two methodologies that we have applied and presented earlier.

For what concerns the commonalities presented in the two methodologies' results we noticed that our approach when navigating the website, especially at a first glance, was rather similar. This was especially true when we compared our experience with the experience of the tester with similar background as ours. We noticed common patterns in elements of confusion or frustrations and especially in expectations.

 This should not surprise us, if we consider that we fit with the user persona profile we picked to carry out the user testing.

It is worth to expand, however, the aspect of expectations: as seasoned users of digital products, we have learned the jargon of such apps, and we intuitively search for patterns we've become familiar with. Such patterns determine our expectations and how fast we learn to interact with a new website. Breaking these expectations leads to frustration and impatience, discouraging us from keeping on using the website.


The main difference resided instead in the additional background knowledge we acquired from our experience as inspectors. Our approach, compared to the one of the users, was far more analytical and took more time in order to find issues and violations of usability guidelines. Regardless, it was specifically our additional technical knowledge that made us judge the website more harshly with respect to the end users.

Our acquired knowledge on heuristics and design principles made it possible for us to pinpoint what exactly caused the users to struggle and make mistakes or to give up altogether.

This is the reason why both methodologies are complementary to each other and should be applied in tandem: usability inspection as experts can create a sort of tunnel vision approach that user testing helps dissipate through the opinions of different people. At the same time inspection techniques give the formality needed to explain user testing results and implement solutions.

\section{Problems discovered}
To summarize, the most critical problems discovered through the Inspection and User Testing phases were:
\begin{itemize}
    \item \textbf{Mismanaged donation function}: UNICEF collects donations for a variety of causes, but finding the donation option for a specific cause becomes a headache;
    \item \textbf{Convoluted way to get back to the main website}: the UNICEF Global button is placed and labeled in such a way that it gets easily missed;
    \item \textbf{Loss of context when moving through domains}: whenever following links of the main website brings you to a completely different domain, this is often not noticeable by the UI alone, which remains the same, but by the lack of the functionalities that were present in the main website;
    \item \textbf{Overwhelming content amount with lack of organization}: the website presents a vast amount of content, be it text, pictures and videos or links to additional content. This material is organized poorly and easily overwhelms the user.
    \item \textbf{Lack of consistency throughout the website}: even between pages or functionalities of the same type there is lack of consistency which betrays the expectations of the user, and useful navigation elements like breadcrumbs are used deceivingly.
    \item \textbf{Malfunctioning research function}: even when the users knows exactly what they're looking for, the research function will mistakenly autocorrect the input and show different results from what expected
\end{itemize}

The severity with which these problems impact on the usability of the website is further confirmed by the SUS resulting score of 45.

\section{Suggestions for redesigns}
Once we identified the issues, we discussed possible solutions that the UNICEF website could put in place to increase their usability.

To tackle the navigation issues, we would suggest the website to implement breadcrumbs for all the pages of the website, instead of just some of them, and to make them more intuitive for the user to use. In the same way, the button to get back to the UNICEF Global website should appear more clear or
be redesigned to be familiar to the user, while also being placed in a position inside the website where it would easily be interpreted as a way to go back. For example, it should be separated from the language and personalization options, and instead located in a top-left corner, with a more descriptive style and name like "Back to UNICEF Global".

For what regards consistency and context, it would be better for articles and posts to have a predefined structure to be applied throughout the website in its complex. While sub-domains and pages that bring you away from the Global website should have a more explicit visual cue to show the change of context, so that the user can recognize that they're not in the main website anymore.

As for donations, while the call for action button on the top right corner gives a direct link to donate to the Agency in general, it doesn't link to other possible causes for donations. Allowing the user to choose what they're donating for from this centralized pages would make it less frustrating than having to browse through the website to find the specific section inside of the content of an article.

The content of articles could be elevated with info-graphs and other visual elements to better breakdown the massive amount of information provided, and it would add to the look of legitimacy of the website.
