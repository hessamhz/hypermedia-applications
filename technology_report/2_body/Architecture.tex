\section{Services used}
SSR was used. % TODO add description

\section{Structure}
As for the structure of the project main folders, we referenced the Nuxt3 standard: the most external index file
contains the code of the homepage, while each page code is inside its own folder, with the same organization used for navigation.
For the pages of the single Service, Project and Person we used 'slug' .....
In order to maintain the same layout throughout the website we have also utilized layouts: a default one with header (navigation bar), body (page content) and footer;
and a 'no-footer' layout, that we used for the chatbot page specifically.


\begin{center}
    \includegraphics[width=0.4\linewidth]{img/folders-structure.png}
    \captionof{figure}{Pages and layouts folders structure}
\end{center}

\vspace{1em}
Available server endpoint with short description

\section{Components}
List of components implemented with description, props and emit (if used)

Here we describe each of the components we implemented in our project:
\begin{itemize}
    \item App:
    \begin{itemize}
        \item \textbf{AppFooter}: the footer present in every page with the contacts, address and map location of The Hive
        \item \textbf{Header}: the navigation bar for medium to large screen
        \item \textbf{Hexagon}: the shape used for various sections regarding the team of The Hive
        \begin{itemize}
            \item \texttt{size} of the hexagon.
        \end{itemize}
        \item \textbf{HexagonalImage}: the component used to shape images into hexagons
        \begin{itemize}
            \item \texttt{size} of the hexagon;
            \item \texttt{src} \textbf{(Required)} URL of the image;
            \item \texttt{alt} Alternative description for the image, if present.
        \end{itemize}
        \item \textbf{Input}: the contact form input field
        \begin{itemize}
            \item \texttt{label} \textbf{(Required)} What information should expect the user to input;
            \item \texttt{name} \textbf{(Required)} The form field name to add to the POST request to the server for the given input;
            \item \texttt{type} The type of input the browser should expect from the user. Defaults to \texttt{text} but it could also be \texttt{'email'};
            \item \texttt{inputClass} Allows the passing of custom CSS class names to the input element, enabling styling customization for the input field;
            \item \texttt{labelClass} Allows for custom CSS class names to be passed to the label element, enabling styling customization for the label.
        \end{itemize}
        \item \textbf{MobileMenu}: the mobile navigation bar for smaller screens
        \item \textbf{SectionHeader}: the gradient or solid header containing the title for all of the pages of the website (excluding homepage and chatbot)
        \begin{itemize}
            \item \texttt{title} \textbf{(Required)} The title of the current webpage to display in the header
            \item \texttt{linkTo} Specifies a navigation target URL. If provided, a clickable link (and breadcrumb) is rendered.
            \item \texttt{backgroundColor} Specifies Tailwind CSS properties for the appearance of the header.
            \item \texttt{color} specifies the color text and other elements of the header. It should just contain the name of the color, which is going to be attached to the Tailwind CSS \texttt{text-} and \texttt{border-} prefixes in the class field.
        \end{itemize}
        \item \textbf{TextArea}: the message input field of the contact form
        \begin{itemize}
            \item \texttt{label} \textbf{(Required)} What information should expect the user to input;
            \item \texttt{name} \textbf{(Required)} The form field name to add to the POST request to the server for the given input;
            \item \texttt{rows} The number of rows the textfield should present. Basically affects the height.
            \item \texttt{teaxtareaClass} Allows the passing of custom CSS class names to the input element, enabling styling customization for the input field;
            \item \texttt{labelClass} Allows for custom CSS class names to be passed to the label element, enabling styling customization for the label.
        \end{itemize}
    \end{itemize}

    \item Icon:
    \begin{itemize}
        \item \textbf{Chevron}:
        \item \textbf{Menu}:
        \item \textbf{Send}:
        \item \textbf{Trash}:
        \item \textbf{X}:
    \end{itemize}
\end{itemize}


\begin{center}
    \includegraphics[width=0.4\linewidth]{img/components-structure.png}
\end{center}
\captionof{figure}{Implemented components}


\section{Extra modules}
Table \ref{table:modules} introduces the modules we used throughout the project:
\begin{table}[h!]
            \centering
            \begin{tabular}{|m{0.3\linewidth}|m{0.6\linewidth}|}
            \hline
            \textbf{Module} & \textbf{Use}\\
            \hline
                 \textit{Tailwind CSS} & CSS framework that enables faster styling and responsiveness without the need to write custom CSS.  \\
            \hline
                 \textit{eslint} & A static code analysis tool that makes code more consistent and avoids bugs by enforcing a set of style and formatting guidelines.  \\
            \hline
            \textit{nuxt-swiper} & Swiper is a modern touch slider with hardware accelerated transitions. It was used to implement a carousel-like presentation of comments/testimonials in the services pages.  \\
            \hline
            \end{tabular}
            \caption{Imported Packages}
            \label{table:modules}
\end{table}

\section{SEO}
Each webpage implements the \texttt{useSeoMeta()} method to define SEO metadata. The metadata injected into each page is:
\begin{itemize}
	\item \textbf{\texttt{Title} (and \texttt{ogTitle})} of the current page and the name of the website;
	\item \texttt{Description} of the current page;
	\item \texttt{ogType} defines the type of content being displayed.
	\item \textbf{\texttt{ogUrl} (and \texttt{canonical})} define the main url of the website.
	\item \texttt{ogSiteName} defines the name of the website.
\end{itemize}

\section{Accessibility}
Accessibility is a fundamental aspect of Web Development and Design and we cared about making our
website as inclusive as possible. For this reason we referenced the WCAG (Web Content Accessibility Guidelines) basic principles and rules
as early as the User Interface Design stage.

\begin{center}
    \includegraphics[width=0.75\linewidth]{img/color-contrast-check.png}
\end{center}
\captionof{figure}{Comparison between early design and newest version}

\vspace{1em}
We mainly focused on color-contrast, alternate-text for images and aria-labels for interactable elements.
We utilized the tools 'WAVE' and 'Lighthouse' in order to check for possible issues and to calibrate better our
implementation as to increase accessibility, especially for the main functionalities of our website.
This perspective allowed us to also increase by extension the general usability.
Of course these tools are extremely useful but not perfect, and we took care of limit cases by using common sense.
\vspace{1em}

\begin{center}
    \includegraphics[width=0.75\linewidth]{img/lighthouse-result.png}
\end{center}
\captionof{figure}{Lighthouse analysis result of the homepage}

\vspace{1em}

\begin{center}
    \includegraphics[width=0.75\linewidth]{img/wave-result.png}
\end{center}
\captionof{figure}{WAVE analysis result of the homepage}
